\chapter{Результаты численного моделирования} \label{ch3:results}

Хорошо известно, что распределения населения и социо-экономической деятельности тяготеют к агрегации и регионализации. Причиной тому может служить наличие многих факторов. Вопрос о том, как человек прибывает в город и желает купить землю для проживания, обсуждается различных точек зрения. В пример можно привести, жаркие споры между экономистами и экологами: экономисты склонны утверждать, что в этом случае речь не идёт ни о каких чувствах, поскольку обсуждаемый индивидуум – экономичный человек, столкнувшийся с двумя необходимостями выбора: насколько большой участок покупать и насколько близко разместиться по отношению к центру города. Экологи отвечают, что индивидуум будет оценивать очевидный характер и расовый состав проживающих рядом людей, качество школ в окрестностях, расстояние до родственников, живущих в городе и тысячи других факторов. 

Развивать модель \textit{взаимодействия} не имеет смысла, описываея ее <<жесткими>> моделями \cite{Arnold}, наличие и учет этих <<тясячей факторов>>, заведомо проигрышный путь (нет никакой возможности учесть все эти факторы). По этой причине, в первой главе в уравнении (\ref{eq1:system}) не было описано характеристики функции \textit{взаимодействия} $F_{s}(u,v)$. Идеализировать модель хотелось бы так, чтобы ее пользователь учитывал только те факторы которые ему необходимы или доступны.

\section{Примеры}
Поставленная задача (\ref{eq1:system}), реализована в виде программного обеспечения, с конфигурируемыми моделями функциями  
на вход которой нужно подать заданные значения <<источника>> $F$, коэффициенты диффузии $K$ и ряд параметров краевых (для каждой из 4 границ, возможно задать различные условия) условий. Результат работы алгоритма был протестирован на модельных примерах из учебника \cite{Samarskiy1987}. Программа реализована в среде разработки MATLAB (студенческая лицензия) версии R2015b.

Рассмотрим пример для системы уравнений состоящих из двух групп. Предпологая, что между обеими группами существует \textit{взаимодействие} в том смысле, что их отношения влияют на характер распределения населения. Отношение могут быть дружелюбными, недружелюбными и <<нейтральными>>. Покажем что разделение и сосуществование зависит от этих \textit{отношений}. Таким образом система уравнений \ref{eq1:system} может быть описана следующим образом:

\begin{empheq}[left=\empheqlbrace]{equation}
    \label{eq3:example1}
    \begin{aligned}
        \frac{\p u}{\p t} = L_{1} u + u \left(a - b u - c v\right) - d_{1} u v \\
        \frac{\p v}{\p t} = L_{2} v + v \left(a - b u - c v\right) - d_{2} u v
    \end{aligned}
\end{empheq}

член $u \left(a - b u - c v\right)$ описывает реакцию населения на экономические условия. Интерпретировать $a$ можно как физическую вместимость городского пространства 

