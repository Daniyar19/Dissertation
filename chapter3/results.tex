\chapter{Результаты численного моделирования} \label{ch3:results}

Поставленная задача, реализована в виде программного обеспечения, с конфигурируемыми модельными функциями  
на вход которой нужно подавать заданные значения <<источника>> $F$, коэффициенты диффузии $K$ и ряд параметров краевых (для каждой из 4 границ, возможно задать различные условия) условий. Результат работы алгоритма был протестирован на модельных примерах из учебника о квазилинейных параболических уравнений \cite{Samarskiy1987}. Программа реализована в среде разработки MATLAB (студенческая лицензия) версии R2015b.

\section{Примеры}

Рассмотрим пример для системы уравнений состоящих из двух групп. Предпологая, что между обеими группами существует \textit{взаимодействие} в том смысле, что их отношения влияют на характер распределения населения. Отношение могут быть дружелюбными, недружелюбными и <<нейтральными>>. Покажем что разделение и сосуществование зависит от этих \textit{отношений}. Таким образом система уравнений \ref{eq1:system} может быть описана следующим образом:

\begin{empheq}[left=\empheqlbrace]{equation}
    \label{eq3:example1}
    \begin{aligned}
        \frac{\p u}{\p t} = L_{1} u + u \left(a - b u - c v\right) - d_{1} u v \\
        \frac{\p v}{\p t} = L_{2} v + v \left(a - b u - c v\right) - d_{2} u v
    \end{aligned}
\end{empheq}
Этот пример был описан Зангом в 1991 году \cite[244]{zhang1991}.
Член $u\left(a - b u - c v\right)$ описывает реакцию населения на экономические условия. Интерпретировать $a$ можно как физическую вместимость городского пространства в точке $(x_{1},x_{2})$. Когда параметр $a$ постоянен (константа), физическая вместимость однародна в пространстве. Если предположить, что $\left(b u + c v\right)$ --- колличественная мера пространства, занимаемого обеими группами, то величину $\left(a - b u - c v\right)$ можно рассматривать как избыток предложения физической вместимостью. Когда эта величина в некоторой точке становится больше нуля, то данное место проживания оказывается более привлекательным для населения. Очевидно, что когда она равна нулю, а члены  $-d_{1}uv$ и диффузионные эффекты пренебрежительно малы, миграция населения прекращается. Член $-d_{1}uv$ служит для измерения \textit{взаимодействия} групп. Коэффициент $d_{1}$ может быть и положительным, и отрицательным, и нулевым. Если он положителен то группе 1 не нравится жить с группой 2. Если $d_{1} = 0$, <<рассовые>> предубеждения отсутсвтуют. Если он отрицателен, высокая плотность группы два притягивает население группы 1. Рассмотрим уравнение\ref{eq3:example1} с конкретными числами:





