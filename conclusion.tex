\chapter{Выводы}

Современный хаос невозможно привести к детерминизму, поэтому развивать модель \textit{взаимодействия} описывая ее <<жесткими>>\cite{Arnold} моделями не имеет смысла. Невилировать это можно лишь описывая ее как <<мягкую>> модель. По этой причине, в первой главе в уравнении (\ref{eq1:system}) не было описано характеристики функции \textit{взаимодействия} . Идеализировать модель хотелось так, чтобы ее пользователь учитывал только те факторы которые ему необходимы или доступны. Но зачастную уменение составлять адекватные математические модели реальных ситуации достаточно не просто. Для чего требуются отдельные часы и дни иследования конкретной проблемы. Но имея <<скелет>> модели, (в моем случае это модель \textit{взаимодействия}), можно решать такие поставленные задачи как (взаимодействие групп населения?). Как сказал В. И. Арнольд  "Успех приносит не только применение готовых рецептов (жестких моделей), сколько математический подход к проблемам явлений реального мира".