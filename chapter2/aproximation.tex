\section{Разностная аппроксимация уравнения}

Система (\ref{eq1:system}) состоит из дифференциальных уравнений одного типа, поэтому для удобства описания численной схемы решения, воспользуемся лишь одним уравнением, с соответствующими краевыми условиями: 

\begin{empheq}[left=\empheqlbrace]{align}
    \label{eq2:diffusion}
    &\frac{\p u} {\p t} = Lu + F, \qquad x \in \Omega, \qquad x = (x_{1},x_{2}) \\
    \label{eq2:cond}
    &\left.\left(A u + K \frac{\partial u}{\partial n} \right) \right|_{\Gamma } =\varphi, \qquad x \in \; \Gamma, \quad A = const.\\
    \label{eq2:init}
    &u(x,t)\vert_{t=0} =u_{0}(x)
\end{empheq}

При написании разностной схемы первым шагом является аппроксимация дифференциального оператора $L = \nabla \left(K \cdot \nabla \right)$. Пусть 
\begin{equation}
        \Lambda y = \Lambda_{1} y + \Lambda_{2} y,
        \qquad \Lambda_{\alpha} y = \frac{\p }{\p x_{\alpha}} \left(\mu \frac{\p y}{\p x_{\alpha}} \right),
        \qquad \alpha = 1,2,
\end{equation}
--- пятиточечный оператор на $\omega_{h}$ с шагами $h_{1}$ и $h_{2}$.
Таким образом дифференциальный оператор $L_{\alpha}$ заменяем разностным $\Lambda_{\alpha}$.

Для численного решения задачи (\ref{eq2:diffusion} ), воспользуемся \textit{неявной} схемой переменных направлений. Эту схему часто называют \textit{схемой Писмена - Рекфорда}\cite{peaceman}. 

\subsection{Метод переменных направлений}
 Дискретизация уравнения (\ref{eq2:diffusion}) проводится на основе локально-одномерной схемы для уравнений с переменными коэффициентами \cite[Самарский,][529]{Samarskiy1977}, которая является абсолютно устойчивой (при любых $\tau$ и $h$) и обладает свойством суммарной аппроксимации имея порядок точности $O(\tau + h^{2})$. Основная идея этого метода состоит в сведении перехода со слоя на слой к последовательному решению одномерной задачи вдоль строк и вдоль столбцов.
 
Введем следующее обозначение: $y(x^{i}_{1},x^{j}_{2},t^{k}) = y^{k}_{i,j}$. И так шаг по времени реализуется в два этапа --- на промежуточном временном шаге (полушаге)  $t = t^{k+1/2}$ проводим дискретизацию двумерного уравнения (\ref{eq2:diffusion}) только в направлении оси $Ox_{1}$ и получаем одномерное уравнение, после его решения вновь проводим дискретизацию уравнения (\ref{eq2:diffusion}), но уже в направлении оси $Ox_{2}$ и, решая полученное одномерное уравнение, определяем значение на целом временном шаге $t = t^{k+1}$.
\begin{empheq}{align}
		\frac{y^{k+1/2}_{i,j} - y^{k}_{i,j}}{0.5\tau} &= \Lambda_{1} y^{k+1/2}_{i,j} + f^{k}_{i,j} \label{eq2:los1},\\[2.5pt]
		\frac{y^{k+1}_{i,j} - y^{k+1/2}_{i,j}}{0.5\tau} &= \Lambda_{2} y^{k+1}_{i,j} + f^{k}_{i,j} \label{eq2:los2}.
\end{empheq}

Коэффициенты диффузии $K = K(u,v)$ зависят от неизвестных функции, поэтому для аппроксимации локально одномерных задач (\ref{eq2:los1}) - (\ref{eq2:los2}) воспользуемся методом баланса (интегро-интерполяционный метод) \cite[571]{Tihonov}.