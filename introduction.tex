\chapter{Введение}
Всякая экономическая деятельность соотносится с определенным временем и местом, а потому важным аспектом эволюционных систем является учет пространственных зависимостей. С прогрессом в условиях транспортировки и связи взаимодействия между различными экономическими переменными становится в значительной мере зависимым от расположения в пространстве. Хорош было бы понять характеристики таких пространственных взаимодействий. В некотором смысле рассмотрим пространственную самоорганизацию, которая отражает свойство изменения структуры и сложное \textit{динамическое} поведение. И можно показать, что медленно текущие градоформирующие процессы могут быть связаны с регулярными и нерегулярными временными осцилляциями.

Проблемы городов весьма усложнились в результате технологического прогресса и изменения поведения людей. Городские системы нашего времени характеризуются возрастанием пространственного и временного разнообразия протекающих в них  процессов. Централизация городов наблюдалась как в развитых странах, так и в странах развивающихся; но вот в некоторых развитых странах начали проявляться процессы децентрализации, и сегодня чаще мы наблюдаем гетерогенные городские образования, нежели гомогенные. Образцами сложности городских форм являются метрополии, такие, как Токио, Париж, Нью-Йорк.

В географии и науке об экономике городов и регионов построено множество моделей для объяснения настоящих и прогноза будущих процессов градоформирования. В современной литературе рассматриваются три основных подхода.

Первый, называемый неоклассической экономикой городов, развивалась экономистами-урбанистами, такими как Алонсо У. и его модель в работе <<Location and Land Use>>\cite{alonso}. Алонсо предположил, что выезд населения за город был связан с двумя факторами — ростом доходов и улучшением транспортной инфраструктуры. Рост доходов в его модели вызывает рост спроса у населения на размер участка, а снижение транспортных издержек снижает стремление жить ближе к центру. Чтобы эти процессы были учтены в модели, Алонсо, первым среди современников, ввел в функции прибыли для компаний и в функции полезности для домохозяйств такие факторы, как размер участка и удаленность участка от центра. Но подход неоклассической экономики ограничен, как правило, анализом равновесных состояний и заведомо предполагает их устойчивость.

Второй подход разрабатывался, в основном, исследователями в области науки о регионах и географии. Время и место в этом подходе играло существенную ролью. Однако, поскольку при этом пространство разбивается на дискретные зоны, оказывается невозможным объяснить внутреннюю структуру таких моделей городов.

Третий подход, называемый пространственным динамическим приближением, для исследования проблем динамики городов использует непрерывное пространство \cite[Занг,][]{zhang1991}. Зангом была предложена такая модель, где в фокусе внимания находится именно проблема эволюции внутренней структуры городов. Таким образом, задача о развитии города описывается системой уравнений в частных производных с соответствующими граничными и начальными условиями. 

Основываясь на результатах исследования в предшествующих трудах, рассматривавших эволюционные подходы к процессам модели градоформирования, в данной магистерской диссертации исследуется взаимодействие переменных в процессе градоформирования, а также пространственная (нелинейная) диффузия участвующая как <<стабилизатор>> или наобарот <<дестабилизатор>> модели.
