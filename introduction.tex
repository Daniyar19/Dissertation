\chapter*{Введение}
\addcontentsline{toc}{chapter}{Введение}
Всякая экономическая деятельность соотносится с определенным временем и местом, а потому важным аспектом эволюционных систем является учет пространственных зависимостей. С прогрессом в условиях транспортировки и связи взаимодействия между различными экономическими переменными становится в значительной мере зависимым от расположения в пространстве. Хорошо было бы понять характеристики таких пространственных взаимодействий или в некотором смысле пространственную самоорганизацию.

Пространственно-временные процессы экономической эволюции отражают свойство изменения структуры и сложное \textit{динамическое} поведение. И можно показать, что медленно текущие градоформирующие процессы могут быть связаны с регулярными и нерегулярными временными осцилляциями.

В географии и науке об экономике городов и регионов построено множество моделей для объяснения настоящих и прогноза будущих процессов градоформирования. В пример можно привести три подхода предложенных еще в прошлом веке специалистами этой области.

Первый, называемый неоклассической экономикой городов, развивалась экономистами-урбанистами, такими как Алонсо У. и его модель в работе <<Location and Land Use>>\cite{alonso}. Алонсо предположил, что выезд населения за город был связан с двумя факторами — ростом доходов и улучшением транспортной инфраструктуры. Рост доходов в его модели вызывает рост спроса у населения на размер участка, а снижение транспортных издержек снижает стремление жить ближе к центру. Чтобы эти процессы были учтены в модели, Алонсо, первым среди современников, ввел в функции прибыли для компаний и в функции полезности для домохозяйств такие факторы, как размер участка и удаленность участка от центра. Но подход неоклассической экономики ограничен, как правило, анализом равновесных состояний и заведомо предполагает их устойчивость.

Второй подход разрабатывался, в основном, исследователями в области науки о регионах и географии. Время и место в этом подходе играло существенную ролью. Однако, поскольку при этом пространство разбивается на дискретные зоны, оказывается невозможным объяснить внутреннюю структуру таких моделей городов.

Третий подход, называемый пространственным динамическим приближением, для исследования проблем динамики городов использует непрерывное пространство \cite[Занг,][]{zhang1991}. Зангом была предложена такая модель, где в фокусе внимания находится именно проблема эволюции внутренней структуры городов. Таким образом, задача о развитии города описывается системой уравнений в частных производных с соответствующими граничными и начальными условиями. 

С тех пор проблемы городов весьма усложнились в результате технологического прогресса и изменения поведения людей. Городские системы нашего времени характеризуются возрастанием разнообразия протекающих в них процессов. Централизация городов наблюдалась как в развитых странах, так и в странах развивающихся; но в городских структурах современности все больше можно увидеть процессы децентрализации, и сегодня чаще мы наблюдаем гетерогенные городские образования, нежели гомогенные. Образцами сложности городских форм являются метрополии, такие, как Токио, Париж, Нью-Йорк. Это хорошо известно, что распределения населения и социо-экономической деятельности тяготеют к агрегации и регионализации. Причиной тому может служить наличие многих факторов. Вопрос о том, как человек прибывает в город и желает купить землю для проживания, обсуждается с различных точек зрения. В пример можно привести, жаркие споры между экономистами и экологами: экономисты склонны утверждать, что в этом случае речь не идёт ни о каких чувствах, поскольку обсуждаемый индивидуум – экономичный человек, столкнувшийся с двумя необходимостями выбора: насколько большой участок покупать и насколько близко разместиться по отношению к центру города. Экологи отвечают, что индивидуум будет оценивать очевидный характер и расовый состав проживающих рядом людей, качество школ в окрестностях, расстояние до родственников, живущих в городе и тысячи других факторов. 

В данной магистерской диссертации, основываясь на результатах исследования в предшествующих трудах, рассматривавших эволюционные подходы к процессам модели градоформирования, исследуется взаимодействие переменных в процессе градоформирования, а также пространственная нелинейная диффузия участвующая как <<стабилизатор>> или наобарот <<дестабилизатор>> модели.
